%----------------------------------------------------------------------------------------
%   USEFUL COMMANDS
%----------------------------------------------------------------------------------------

\newcommand{\dipartimento}{Dipartimento di Matematica ``Tullio Levi-Civita''}

%----------------------------------------------------------------------------------------
% 	USER DATA
%----------------------------------------------------------------------------------------

% Data di approvazione del piano da parte del tutor interno; nel formato GG Mese AAAA
% compilare inserendo al posto di GG 2 cifre per il giorno, e al posto di 
% AAAA 4 cifre per l'anno
\newcommand{\dataApprovazione}{Data}

% Dati dello Studente
\newcommand{\nomeStudente}{Nicolò}
\newcommand{\cognomeStudente}{Trinca}
\newcommand{\matricolaStudente}{2011070}
\newcommand{\emailStudente}{nicolo.trinca@studenti.unipd.it}
\newcommand{\telStudente}{+ 39 392 74 20 300}

% Dati del Tutor Aziendale
\newcommand{\nomeTutorAziendale}{Paolo}
\newcommand{\cognomeTutorAziendale}{Pietrobon}
\newcommand{\emailTutorAziendale}{risorseumane@replay.it}
\newcommand{\telTutorAziendale}{+ 39 042 39 251}
\newcommand{\ruoloTutorAziendale}{}

% Dati dell'Azienda
\newcommand{\ragioneSocAzienda}{FASHION BOX S.p.A}
\newcommand{\indirizzoAzienda}{Via Marcoai, 1 – Asolo (TV)}
\newcommand{\sitoAzienda}{www.replayjeans.com }
\newcommand{\emailAzienda}{mail@mail.it}
\newcommand{\partitaIVAAzienda}{P.IVA IT03676290269}

% Dati del Tutor Interno (Docente)
\newcommand{\titoloTutorInterno}{Prof.Lamberto Ballan}
\newcommand{\nomeTutorInterno}{Lamberto}
\newcommand{\cognomeTutorInterno}{Ballan}

\newcommand{\prospettoSettimanale}{

    \textbf{Inizio 10/05/2023}\\
    \textbf{Esclusione dei Venerdì: 12/05/2023, 19/05/2023 e 26/05/2023} \\
    \textbf{Fine 07/07/2023}
    % Personalizzare indicando in lista, i vari task settimana per settimana
    % sostituire a XX il totale ore della settimana
    \begin{itemize}
        \item \textbf{Prima Settimana - Ambientamento e analisi del problema (32 ore)}
              \begin{itemize}
                  \item Ambientamento azienda FASHION BOX S.p.A;
                  \item Analisi problema e obiettivi;
                  \item Presa visione dell’infrastruttura esistente;
              \end{itemize}
        \item \textbf{Seconda Settimana - Studio e documentazione (32 ore)}
              \begin{itemize}
                  \item Documentazione sull argomento GAN;
                  \item Analisi sull utilizzo di GAN per la generazione d immagini;
              \end{itemize}
        \item \textbf{Terza Settimana - Studio e documentazione (32 ore)}
              \begin{itemize}
                  \item Studio sulle implementazioni GAN;
              \end{itemize}
        \item \textbf{Quarta Settimana - Setup ambiente e framework (40 ore)}
              \begin{itemize}
                  \item Documentazione sull argomento GAN;
                  \item Valutazione dei frameworks e delle tecnologie per l'implementazione di GAN;
                  \item Installazione framework;
              \end{itemize}
        \item \textbf{Quinta Settimana - Preparazione dataset (40 ore)}
              \begin{itemize}
                  \item Raccogliere un dataset appropriato per l'addestramento del GAN;
                  \item Preprocessare il dataset per adattarlo al formato richiesto dal framework;
              \end{itemize}
        \item \textbf{Sesta Settimana - Sviluppo (40 ore)}
              \begin{itemize}
                  \item Sviluppo POC;
              \end{itemize}
        \item \textbf{Settima Settimana - Sviluppo (40 ore)}
              \begin{itemize}
                  \item Sviluppo POC;
                  \item Test del modello GAN addestrato;
              \end{itemize}
        \item \textbf{Ottava Settimana - Conclusione Sviluppo (40 ore)}
              \begin{itemize}
                  \item Sviluppo POC;
                  \item Srittura documentazione;
                  \item Effettuare eventuali ottimizzazioni sulla rete generatrice e discriminatrice;
                  \item Valutare le prestazioni del modello GAN addestrato;
              \end{itemize}

        \item \textbf{Nona Settimana - Conclusione (16 ore)}
              \begin{itemize}
                  \item Srittura documentazione;
                  \item Presentazione lavoro
              \end{itemize}
    \end{itemize}
}



% Indicare il totale complessivo (deve essere compreso tra le 300 e le 320 ore)
\newcommand{\totaleOre}{312}

\newcommand{\obiettiviObbligatori}{
    \item \underline{\textit{O01}}: Comprendere il funzionamento teorico dei GAN e le loro applicazioni per la generazione di immagini.;
    \item \underline{\textit{O02}}: Sviluppare la capacità di autogestirsi e affrontare argomenti sull'argomento GAN attraverso l'autoformazione;
    \item \underline{\textit{O03}}: Sviluppo di un POC dimostrativo;

}

\newcommand{\obiettiviDesiderabili}{
    \item \underline{\textit{D01}}: Implementazione di un modello GAN per la generazione di immagini funzionante;
    \item \underline{\textit{D02}}: Generazione di immagini realistiche e coerenti con il dataset di addestramento
    \item \underline{\textit{D03}}: Personalizzare il modello GAN per l'inserimento del brand "Replay" nelle immagini di output
}

\newcommand{\obiettiviFacoltativi}{
    \item \underline{\textit{F01}}: Ottimizzazione del modello GAN
    \item \underline{\textit{F02}}: Valutare le prestazioni del modello GAN
    \item \underline{\textit{F03}}: Test applicazione prodotto finale in ambito aziendale
}